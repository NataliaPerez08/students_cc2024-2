\documentclass{article}
\usepackage{graphicx}

\title{Práctica 1:  \\ Cómputo Concurrente 2024-2}
\author{Pérez Romero Natalia Abigail}
\date{\today}

\begin{document}
\maketitle

\section{Introducción y jugando con hilos}

Modifica el código de Hilos.java, realizando lo siguiente:
\begin{enumerate}
    \item Genera una estructura de datos (en donde después guardarás los hilos).
    \item Agrega 10 hilos a esta estructura de datos mediante un for (no te olvides de inicializarlos).
    \item Finalmente, haz join a cada hilo con un for o foreach.
    \item Agrega una captura de pantalla a tu reporte sobre como quedó tu código al final
\end{enumerate}

\section{Synchronized}
¿Qué significara? Creo que un sección del programa que provoca que un hilo se ejecute antes de otro, según indiquemos

Synchronization en Java:

Los hilos se comunican principalmente compartiendo el acceso a los campos y a los objetos a los que se refieren los campos de referencia. Esta forma de comunicación es extremadamente eficiente, pero posibilita dos tipos de errores: interferencia entre hilos y errores de consistencia de memoria. La herramienta necesaria para evitar estos errores es la sincronización.

Sin embargo, la sincronización puede introducir la contención de hilos, que se produce cuando dos o más hilos intentan acceder simultáneamente al mismo recurso y hacen que el tiempo de ejecución de Java ejecute uno o más hilos más lentamente, o incluso suspenda su ejecución. La inanición y el livelock son formas de contención de hilos.

Synchronization. (n.d.). Retrieved from https://docs.oracle.com/javase/tutorial/essential/concurrency/sync.html

\section{Cuestionario}

\begin{enumerate}
    \item ¿Porque se utiliza Interrupted Exception en el método main?
    
    \item ¿Para que sirve el método Join?
    
    \item ¿Qué pasa si no le hacemos Join a los hilos?
    
    \item ¿Cuáles son las ventajas de implementar Runnable contra extender de Thread?
    
    \item ¿Cuál es la diferencia de implementar Runnable contra Callable?
    
    \item ¿Se puede predecir el orden en el que se imprime el mensaje de la clase Hilos?
    
    \item En el archivo Hilos2.java, ¿Qué pasa si sacamos la instancia de la clase “h” de t1, es decir, poner h antes de declarar t1?
    
    \item Escribe que variables son locales (variables que estan en memoria del hilo) y que variables son compartidas de cada archivo y el por qué. Puedes tomarle captura al código y encerrar en un recuadro dichas variables.
    
    \item \textbf{Contesta las preguntas de la sección de Synchronized}
    \item ¿Cómo podriamos darle un comportamiento diferente a los hilos?
    \item Escribe lo aprendido sobre esta práctica, así como tus conclusiones. También comparte si tuviste dificultades y los descubrimientos o alguna cosa de interés.
\end{enumerate}
\end{document}
